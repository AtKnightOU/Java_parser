\documentclass[12pt,a4paper]{report}
\usepackage[latin1]{inputenc}
\usepackage{amsmath}
\usepackage{amsfonts}
\usepackage{amssymb}
\usepackage{url}

\begin{document}
	\begin{center}
		\textbf{Top Directory: cupJFlex/java1.2Parser}
	\end{center}

\begin{flushleft}
\begin{tabular} {l p{12cm}}
Directory: & src/main/cup \\
File:      & java12.cup \\   
Inhoud:    & mijn grammar en actions voor Java \\
\ \\
Directory: & src/main/jflex  \\
File:      & java.flex  \\  
Inhoud:    & mijn lex-file voor java \\
\ \\
Directory: & cupJFlex/java1.2Parse \\
File :     & Make \\
Doel:      & Roept jflex en cup aan en genereert sym, lexer en parser in out/genfiles. \\
           & Maakt gebruik van common/Makefile.inc \\
\ \\
Directory: & cupJFlex/java1.2Parse \\
File:      & runStandaardTest  \\  
Doel:      & Gaat na of grammar en lex-file nog correcte uitvoer kan genereren voor 
             Test.java == Dashboard. \\
           & \\
\ \\
Directory: & cupJFlex/java1.2Parse \\
File:      & genereerTemplates  \\  
Doel:      & Genereert voor \'e\'en (Java) file the body van een template file: \\
		   & class name, attributen, operations, parameters, edges. \\
Aanroep:   & genereerTemplates Java file \\
\textbf{to do}:   & Script schrijven waarmee de uitvoer aangepast wordt. \newline
                    Toevoegen: $<$templates$>$ $<$template name="Sourcecode"$>$ ... $<$/template$>$ $<$/templates$>$. \newline
                    Voor $<$/template$>$ moeten alle $<$edges$>$ komen te staan. \newline
                    \textbf{edges voor associaties en dependencies genereren.}
                    Nog uitzoeken: lokatie van repeating group en peninsula.
\ \\
Directory: & cupJFlex/java1.2Parse \\
File:      & genereer\dots\dots Templates  \\  
Doel:      & Genereert voor Java-files in een Netbeans-projectde body van een template file: \\
& class name, attributen, operations, parameters, edges. \\
Aanroep:   & genereer\dots\dots Templates \\
\textbf{to do}:   & Script schrijven waarmee de uitvoerfile aangepast wordt. \newline
Toevoegen: $<$templates$>$ $<$template name="Sourcecode"$>$ ... $<$/template$>$ $<$/templates$>$. \newline
Voor $<$/template$>$ moeten alle $<$edges$>$ komen te staan. \newline
\textbf{edges voor associaties en dependencies genereren.}
Nog uitzoeken: lokatie van repeating group en peninsula.
\ \\
\ \\
Websites:  & \url{http://www2.cs.tum.edu/projects/cup/}  \newline  \url{https://www.jflex.de} \\
\end{tabular}
\end{flushleft}
\pagebreak
\begin{center}
	\textbf{Top Directory: NetBeansProjects/DP\_detection\_in\_sources}
\end{center}

\begin{flushleft}
\begin{tabular} {l p{12cm}}
Directory:   & \\
File:        & invoerSysteem.xml \newline 
               Door aanroep van \emph{genereerTemplates}, zie boven,  kan voor elke Java-file
               kan deze template file worden samengesteld. \\
Netbeans:    & args[0] is ingesteld door: \newline
               Rechtermuisklik op \emph{DP\_detection\_in\_sources},
               klik op \emph{Set Configuration}, klik op  \emph{Customize}. \\
\ \\
Directory:   & \\
File:        & templates.xml  \newline
               Bevat de beschrijving van 25 DP's m.b.v. repeating groups, peninsula. \newline
               De design patterns Facade en Factory Method komen beide twee keer voor. \\
\ \\            
Package:     & DP\_Templates\_Def\_Parser \\
alle files   & Alle Java sources worden gebruikt door de SAX-parser om een XML-file met \newline
               hetzij  GoF\-templates  hetzij invoerSysteem.xml te parsen. \newline
               Het result van de parsing wordt bewaard in Java files van het package 
               \emph{DP\_Templates\_Def}.
\end{tabular}
\end{flushleft}
\end{document}